\begin{titlepage}
    \begin{center}
    	%\thispagestyle{firstpage} % Apply the page style for the first page (no headers and footers)
    	\pagenumbering{gobble}
        \vspace*{1cm}
        \maketitle % Print the title
        
        
        \lettrineabstract{The specificity of DNA base-pair interactions gives considerable functional control in the design of anisotropic nano-particles, enabling the formation of a range of liquid crystal phases. Initial benchmarked tests confine the critical volume fraction of the nematic phase transition for linear ds-DNA mesogens (aspect ratio 10) to within $0.39<\phi<0.44$, in agreement with the theoretical value of $\phi = 0.4$.  I then introduce a novel `nunchuck' mesogen - two rigid rods of ds-DNA connected via a flexible linker of ss-DNA, and demonstrate the existence of an entropy\textendash driven phase transition to a quasi-nematic ($S_{n} = 0.58$) phase in this system. I outline further evidence for the existence of more exotic `herringbone', twisted-nematic and biaxial-nematic phases through the use of the pair-wise orientational correlation function. Finally, I present an alternative method for phase identification through the study of dynamic properties, and identify the formation of nematic and smectic phases through the measurement of directional diffusion coefficients.}
        
%        \begin{figure}[h!]
%        	\centering
%        	\includegraphics[width=0.4\linewidth]{Figures/university-of-cambridge-seeklogo.com}
%        	\caption*{}
%        \end{figure}
        
        
        \vspace*{5cm}
        
        Supplementary Material: \textit{If necessary} \\
		Online Lab-book: \textit{To be included} \\
		
		\vspace*{2cm}
%		
		Except where specific reference is made to the work of others, this work is original and has not been already submitted either wholly or in part to satisfy any degree requirement at this or any other university.
        
       
        
    \end{center}
\end{titlepage}